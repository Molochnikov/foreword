\documentclass[12pt]{article}
\usepackage[utf8]{inputenc}
\usepackage[a6paper]{geometry}
\usepackage[LGR,T2A]{fontenc}
\usepackage[russian]{babel}
\usepackage{indentfirst}
\usepackage{tocloft}
\usepackage{fancyhdr}
\usepackage{calc}
\usepackage{xcolor}
\usepackage{tikz}
\usepackage{titling}
\usepackage{hyperref}
\usepackage{titlesec}


\newcommand{\rom}[1]{\uppercase\expandafter{\romannumeral #1\relax}}

\title{Предисловие к первому изданию собрания сочинений}
\author{Людвиг Андреас фон Фейербах}
\date{}

\tolerance=10000
\hbadness=10000
\vbadness=10000
\clubpenalty=10000
\widowpenalty=10000
\displaywidowpenalty=10000
\interfootnotelinepenalty=10000

\begin{document}

\maketitle

\newpage

Открывая этим томом собрание своих сочинений, 
я прежде всего должен заметить, что эта коллекция 
древностей обязана возникновением не мне, но моему 
издателю.

\begin{quote}

Je ferme \`a jamais 

Ce livre \`a ma pens\'ee \'etranger d\'esormais, 

Je n'\'ecouterai pas ce qu'en dira la foule, 

Car qu'importe \`a la source o\`u son onde s'\'ecoule?

\end{quote}


Так думал я не только по поводу той небольшой  
брошюры, в которой привел эти выразительные слова  
одного французского поэта, то же самое приходило мне в голову и при завершении работы над каждым моим 
произведением. Как только была готова какая-либо 
книга, я говорил ей: прощай навсегда; каждая книга 
заставляла меня лишь осознать мои собственные  
ошибки и изъяны и поэтому оставляла лишь настоятельное желание погасить воспоминание о ней новой книгой. 
И вдруг мне было предъявлено требование обратить 
свой неудовлетворенный, противный священному  
писанию, противобиблейский взор на все давно уже  
исчезнувшие из моей памяти собственные писания. Что  
делать? Плыть против течения жизни? Вопреки  
требованиям природы идти вспять, а не вперед? Вопреки 
хорошему вкусу пережевывать давно переваренное? 
Вопреки телесному инстинкту воскрешать мертвых, 
вместо того чтобы рожать детей? Нет, любезный господин Виганд, это противоречит моей натуре, это  
противоречит моему чувству. 

Между тем, как это часто бывает в жизни,  
размышление взяло наконец верх над противящимся чувством. 
Я так рассуждал и спорил с самим собой. Разумеется, 
взгляд на твои прежние писания для тебя  
нерадостный взгляд на твое прошлое, давно уже ставшее тебе 
чуждым. Но если для тебя нечто является прошлым, 
оказывается ли оно прошлым также и для других? 
Разве та пелена, которая спала с твоих глаз, не  
является в настоящее время доспехами твоих противников? 
Возьмем безмозглых философов, лишенных  
чувственных материальных основ своих мыслей, разве они при 
слове <<мясо>> не думают о паштете из гусиной печенки, разве под своей чувственностью как свидетельницей 
истинности они не подразумевают свои семенные  
яички, разве при упоминании зрительного бугра (thalamus 
nervorum opticorum) 2 они не помышляют о брачном 
ложе? Такие философы не напоминают ли тебя самого, 
каким ты некогда был в качестве студента и доцента, 
занимающегося философией Гегеля, Декарта или  
Спинозы? Разве ты не лишил себя кредитоспособности 
именно благодаря своим позднейшим писаниям,  
которые выражают твои теперешние взгляды и мысли,  
впрочем, увы! В весьма несовершенной форме? Разве ты 
ими не подорвал надежд, возлагавшихся на твои  
прежние писания? Правда, это делалось весьма близоруко. 
Но ведь и твое тусклое прошлое, лежащее позади  
твоего писательского поприща, продолжает оставаться  
актуальным. Разве правоверные и благонадежные  
богословы, которые хотят обработать тебя, зрелого  
человека, продвинулись вперед по сравнению с тобой, когда 
ты был благочестивым гимназистом? Разве в то время 
Библия не была для тебя величайшим авторитетом,  
источником правды, словом божьим? Разве ты не  
доказывал твоим погруженным в сомнения школьным  
товарищам объективности, подлинной реальности богочеловека, 
хотя для тебя бессознательно и разум был уже  
авторитетом, а теперь для тебя богочеловек лишь любовное 
чадо сверхприродного, сверхчеловеческого блаженства? 
Разве ты не поступил в университет в качестве  
позорно 
ной памяти схоласта-богослова, т. е. богослова,  
желавшего в образах веры усмотреть истины разума? Разве 
ты в свое время не думал, что с утратой веры ты  
потерял бы также связь, объединяющую тело и душу,  
основу и опору твоей жизни? Но разве такая вера не есть 
и в наши дни повсеместная вера? Не слышал ли ты 
сам из уст министров и парламентариев, что  
религиозная вера есть основа человеческого существования и 
благополучия? О, ты был бы великим мыслителем, если 
бив наши дни рассуждал так, как рассуждал некогда, 
будучи богобоязненным школьником!

Всего этого нельзя отрицать; но разве  
современность --- мерило истинности и пригодности для  
человечества? Разве она законодательница будущего? Разве 
даже для ближайшего будущего не оказывается  
истиной то, что в настоящее время признается  
заблуждением? Разве для этого будущего не становится делом 
практики то, что сейчас имеет значение лишь теории? 
Должен ли взгляд на нынешнее положение вещей  
сковывать твой неустанно стремящийся вперед дух? Ни 
в коем случае. Ты имеешь право оживить свое прошлое 
лишь тогда, когда ты сам с этим прошлым примирился, 
когда ты в состоянии совмещать его с настоящим, с 
своей нынешней точкой зрения.

Итак, окинь непредвзятым взором свое прошлое, и 
ты увидишь, созвучно ли оно и в какой степени твоему 
настоящему. Учти прежде всего тот способ, каким ты 
высказывался в своих писаниях, даже самых ранних. 
Высказывался ли ты как абстрактный философ? Нет! 
Ты мыслил как философ, но писал не как философ; 
ты неизменно превращал выражаемое тобой существо 
мысли в телесное, полнокровное существо. К объекту 
мысли ты предъявлял требование, чтобы он  
одновременно был и объектом эстетики; ты знал, что  
философия, как таковая, чистый разум, чистая мысль --- ничто 
для человека и никак не может на пего воздействовать, 
что человека можно убедить в истине лишь в том  
случае, если ее из мысленной сущности, из сущности  
умозрительной (ens rationis) превратить в сущность,  
подобную человеку, в чувственную Сущность. Исходя, правда, не только из сознания этого, но и из  
внутренней необходимости, ты уже в первом анонимном  
произведении 3 высказал поэтому свои мысли о смерти и бессмертии только поэтическим, т. е. чувственным,  
языком. Прозой написано лишь предисловие к этому  
произведению, текст его стихотворный. Что в предисловии 
высказано как философская истина, в самом тексте  
выражено в виде религиозной, т. е. антропологической, 
истины, как предмет ощущения, как непосредственная 
очевидность. В этом одном значение указанного  
произведения и отличие его от других текстов, отрицающих 
бессмертие, появившихся почти одновременно с ним; 
лишь этим и определяется первое резкое различие  
христианского и нехристианского взгляда на жизнь; 
пусть это единственный, но очень существенный пункт; 
ведь только там в истории человечества образуются  
перерывы в старом и зачатки новой жизни, где неверие в божества старого мира проявляется в виде безусловного 
убеждения, как лично прочувствованная истина, как 
осязаемая уверенность.

Этим же предметом ты был занят и в позднейшие 
годы, но исходя уже не с точки зрения  
пантеистического тождества, а с точки зрения политеистического 
различия, с точки зрения принципа Лейбница,  
различия определенности --- таковы твои <<Юмористически- 
философские афоризмы>> 4. Говоря кратко, мысль  
данного произведения такова: дух, человеческая душа не есть та неопределенная, имматериальная, простая,  
абстрактная сущность, над которой психологи ломают 
себе голову, --- это не что иное, как существенная  
определенность человека, делающая его тем, что он есть, это 
характерное своеобразие, выразительнейшая форма его 
индивидуальности. Но как ты выразил эту мысль со 
всеми ее последствиями? Символически, образно, т. е. 
конкретно, фактически на одном определенном  
примере, который вместе с тем вполне реализует и  
конкретизирует указанную общую мысль *. Этот чувственный, 
* Впрочем, политическое состояние Германии оказало 
большое, далеко не отрадное влияние на форму, даже на содержание, вообще на стиль мною написанного. Этот стиль конкретный способ созерцания и изложения ты  
применял всюду, даже в области критики и истории  
философии; всюду ты абстрактное связывал с конкретным, нечувственное с чувственным, логическое с  
антропологическим. Отличие нынешних твоих писаний от прежних 
в том, что ты превратил в сущность то, что для тебя 
раньше было лишь образом, превратил в  
действительное, в само содержание то, что для тебя ранее было 
лишь формой; теперь ты сознательно, открыто  
высказываешь то, что раньше высказывалось тобой  
косвенно, неосознанно. В прежнее время ты высказывался или 
мыслил по меньшей мере вразрез с официальной  
философией: подлинна та философия, которая сама себя 
упраздняет, которая обнаруживает себя не как  
философия, которая по форме, по виду не есть философия; 
теперь же ты прямо говоришь: подлинная философия 
есть отрицание философии, вовсе не есть философия; 
прежде ты думал, а также и высказывал, хотя не  
формально, не словесно, а фактически: истинное должно 
быть наличным, действительным, чувственным,  
наглядным, человеческим; теперь ты вполне последовательно 
выражаешься обратным способом: только  
действительное, чувственное, человеческое есть истинное.

Взгляни на содержание своих произведений, в  
особенности исторических, в которых ты под чужими 
именами высказывал собственные мысли: бросается в 
глаза связь этих книг с теми, которые характеризуют 
твою теперешнюю точку зрения; таков «Бейль»5,  
которого ты, следуя поставленной себе задаче, писал с позиций рационализма и который этим существенно  
отличается от твоей «Сущности христианства»6, впрочем, рационалистическая точка зрения повлияла и на эту 
последнюю книгу вразрез с ее подлинным духом; таков 
же и твой «Лейбниц» 7, в котором ты рассмотрел и  
выдвинул в противоположность принципу тождества  
(выраженному в твоих «Мыслях о смерти» и  
господствующему еще и в первом томе твоей «Истории философии») 
не столько нагляден, сколько часто затемняет мысль.  
Впрочем, здесь я ограничиваюсь лишь беглым наброском хода 
своих мыслей. 
39 
принцип различия, индивидуальности, чувственности, 
но выдвинул номиналистически, абстрактно, выразил 
не в чувственных формах и даже вразрез с ними;  
таким образом, ты дал критику богословия, правда  
одностороннюю, опирающуюся на метафизику. Остается 
только пробел в виде первого тома твоеіі истории  
философии. Тут особенно важную роль играет отношение 
бытия к мышлению, поводом к чему явилось положение 
Декарта «мыслю, следовательно, существую» и так  
называемое онтологическое доказательство бытия бога 
как высшей мыслящей сущности. 
Всякого рода верующие люди с давних пор муча- 
лись над доказательствами бытия божьего и  
утверждали, что бытие бога нельзя доказывать, да оно и не 
нуждается в доказательствах, будучи непосредственно 
очевидным. Но это утверждение находится в  
противоречии как с историей, так и с разумом. В отличие от 
самоочевидности человека непосредственно очевидно 
только бытие природы, но не бытие бога, т. е.  
существа, отличного от природы и человека. Во всяком  
случае первоначально это существо скорее опирается лишь 
на один вывод, а именно на тот, что природа не может 
существовать сама по себе, что она предполагает  
другое существо, таким образом, это существо оказывается 
существом, не подлежащим сомнению. Поэтому ты  
правильно сделал, что не отнесся легкомысленно к вопросу 
о существовании бога. В особенности тебя занимал  
вопрос о природе, свойствах этого существования. Бог есть 
сущность, данная лишь разуму, мысли, абстракции от 
чувственного; у него нет ни одного свойства  
чувственного существа. К чему же сводится бытие такой  
сущности? Может ли существование нечувственной  
сущности быть чувственным? Как это возможно? Ведь бытие 
не может отличаться от сущности. Ты утверждал так: 
«Как сущность, так и существование бога есть дело  
разума». «Существование бога нельзя отличить от его 
сущности: следовательно, хотя его существование  
сущностное, нечувственное у но, чтобы убедиться в этом  
существовании, я нуждаюсь в другом органе, нежели  
разум». Что другое это значит, как не то, что разумная 
40 
сущность имеет лишь разумное существование? И 
опять-таки, какой иной смысл этого положения, кроме 
следующего: бог, как нечувственная, лишь мысленная 
сущность, не существует вне разума? В самом деле, 
ведь существование, отличное от разума, или  
существование вне разума есть лишь чувственное  
существование. Отсюда явствует, как прост был переход к первой 
главе «Сущности христианства», в которой сказано: бог, 
как нечувственное, абстрактное, нечеловекообразное 
существо, есть не что иное, как сущность разума! Все 
же к этому результату ты пришел лишь спустя семь 
или восемь лет, во всяком случае с полной ясностью и 
определенностью. Что же тебя так долго  
задерживало, почему из недостатка чувственного существования 
ты не заключал об отсутствии существования вообще? 
Почему реальным, действительным существом для тебя 
была простая мысленная сущность? Это потому, что  
вообще мысль для тебя была сущностью, предмет мысли, 
как таковой, представлялся действительным,  
субъективное — объективным, мышление — бытием. Где 
мысль, как таковая, принимается за истину и реальность, 
там, естественно, не возникает сомнений в истинности 
и реальности существа, выражающего не что иное, 
как сущность мышления, составляющего не что иное, как 
кульминационный и центральный пункт абстракции. 
Почему же для тебя вообще мысленная сущность была 
сущностью реальной? Потому что ты еще не усвоил 
смысла и значения сущности чувственной, потому что 
для тебя подлинно реальная, чувственная сущность 
была лишь конечной, пустой и ничтожной сущностью. 
Где действительность кажется нереальной, там  
нереальное неизбежно представляется действительностью. 
Итак, исключительно недооценка истинности л  
существенного значения чувственности служила рубежом 
между твоей прежней позицией и теперешней. Как же 
ты пришел к этим взглядам? Как они у тебя возникли — 
посредством произвольного зарождения (generatio aeqi- 
voca) или органическим путем? Органическим. Уже в 
этом первом томе находятся зародыши указанных  
взглядов. Сколько ты ни полемизировал но поводу учения 
41 
о происхождении идей и по другим пунктам с  
основоположниками эмпиризма Бэконом, Гоббсом, Гассен- 
ди, все же ты изучал их с особой любовью, прежде 
всего Бэкона, объявив уже тогда эмпиризм «делом  
философии». И если ты не сразу сделал выводы из того 
значения, которое сам придал опыту, то этому  
препятствовала лишь природа рассматриваемых тобой  
объектов. Ты нуждался во времени и пространстве, чтобы 
чувственными методами исследовать чувственные вещи 
и сущности, чтобы приобрести научную уверенность в 
реальности чувственно данного; к счастью, тебе  
представилась эта возможность. Однако вместе с тем сама 
эта уверенность прежде всего была только  
естественнонаучной. Ведь можно признавать истинность  
чувственного в области естественнонаучной и все-таки  
отрицать ее в философской и религиозной области; можно 
даже одновременно быть материалистом и  
спиритуалистом, светским вольнодумцем и религиозным  
обскурантом, практически атеистом, а теоретически — 
правоверным теистом. Бэкон, Декарт, Лейбниц, Бейль, 
вообще новое и новейшее время — блестящий пример 
этого противоречия. Как же ты преодолел указанное 
противоречие? Каким образом от естественнонаучной 
реальности чувственного ты пришел к его абсолютной 
реальности? Только признанием, что существо, которое 
противопоставляется чувственности как нечто  
чужеродное, само есть не что иное, как абстрактная или  
идеализированная сущность чувственного. К этим взглядам 
ты пришел прежде всего в области религии, поэтому ты 
полемизировал против философии, утверждавшей, что 
ее содержание совпадает с содержанием религии, что 
она только упраздняет чувственную форму, в которую 
религия погружает это содержание; ты возражал: эту 
форму нельзя оторвать от содержания религии, не  
упразднив самой религии, ибо она сущность религии. Но 
то, что ты признал за существенную сторону религии, 
то еще первоначально не было существенным, по  
крайней мере теоретически, для тебя, для твоего сознания, 
не было твоим знанием; у тебя еще было неладно в  
голове от абстрактной сущности разума, от сущности  
философии в отличие от действительной чувственной сущ- 
42 
ности природы и человечества. Не без этого  
противоречия была написана твоя «Сущность христианства», во 
всяком случае частично. Противоречие это было  
действительно преодолено только в твоем «Лютере» 8,  
которого поэтому нельзя назвать лишь «материалами», как 
это значится в заглавии, ведь вместе с тем он имеет 
самостоятельное значение. Лишь в «Лютере» ты  
окончательно «стряхнул» с себя философа, решительно  
превратив философа в человека. 
Итак, вот как связаны между собой твои  
произведения: они содержат историю непроизвольного  
возникновения и развития, а следовательно, и оправдания твоей 
теперешней точки зрения. Но может быть, твоя  
современная точка зрения устарела? Ты заявил, что  
настоящее еще не определяет тебя, но ты, очевидно, здесь  
прибег к синекдохе, приняв за целое часть настоящего, ту 
часть, которая помешана на сохранении или даже  
реставрации минувшего. Следовательно, нужно выслушать 
и противную партию. Чего же она хочет? Политических 
и социальных реформ. Но она ни в коей степени не 
заинтересована в делах религии и еще меньше —  
философии. Для этой другой партии религия есть  
совершенно безразличное или даже давно решенное дело; 
представители ее заявляют, что теперь речь идет не о 
бытии или небытии бога, но о бытии или небытии  
людей, не о том, составляет ли бог с нами единую или 
разнородную сущность, а о том, равны или неравны мы, 
люди, между собой, не о том, как человек находит  
справедливость у бога, но как он ее находит у людей, не о 
том, вкушаем ли мы вместе с хлебом тело господне, но 
о том, чтобы у нас был хлеб для наших собственных 
тел, не о том, что мы воздаем богу божье, кесарю  
кесарево, но о том, чтобы мы наконец воздали человеку  
человеческое, не о том, являемся ли мы христианами или 
язычниками, теистами или атеистами, но о том, что мы 
люди, и именно телом и душой здоровые, свободные, 
деятельные и жизнеспособные люди или будем  
таковыми. Простите, господа! Как раз этого домогаюсь а я 
сам. Кто обо мне не говорит и не знает ничего  
большего, кроме того, что я —атеист, тот вообще ничего не 
43 
говорит и ничего обо мне не знает. Вопрос о том,  
существует ли бог или нет, как водораздел между теизмом и 
атеизмом достоин семнадцатого и восемнадцатого, но 
отнюдь не девятнадцатого века. Я отрицаю бога; для 
меня это значит: я отрицаю отрицание человека, я  
утверждаю чувственное, истинное, следовательно,  
неизбежно также политическое, социальное место человека 
взамен иллюзорного, фантастического, небесного  
пребывания человека, которое в действительной жизни  
неизбежно превращается в отрицание человека. Для меня 
вопрос о бытии или небытии бога есть лишь вопрос о 
бытии или небытии человека. 
Пусть это верно, но ведь твоя тема все же остается 
лишь вопросом головы и сердца, зло же пребывает не 
в голове и сердце, его место в желудке человечества. 
Какую пользу окажет ясность и здоровое состояние  
головы и сердца, если желудок не в порядке, если основа 
человеческого существования повреждена? «Я  
чувствовала, как злые мысли поднялись у меня из  
желудка», — сказала одна преступница. Эта преступница — 
зеркало современного человеческого общества. Одним 
дано все, чего бы ни захотела их изысканная алчность; 
у других нет ничего, нет даже самого необходимого в 
желудке. Отсюда всякое зло и страдание, отсюда все 
болезни головы и сердца человечества. Поэтому все, что 
не нацелено непосредственно на познание и устранение 
этого основного зла, есть бесполезный хлам, и к этому 
хламу принадлежат твои произведения в целом и в  
отдельности. Увы5 увы! Однако ведь много зла, даже  
страданий желудка, которые коренятся только в голове; и 
я поставил себе задачей основательное исследование 
и исцеление головных и сердечных болезней  
человечества: тому были внутренние и внешние причины. То 
же, что человек поставил себе задачей, то нужно tenax 
propositi — твердо проводить в жизнь; то, чему он  
положил начало, следует выполнить основательно,  
самоотверженно. Поэтому и на издание собрания  
сочинений я решился лишь при условии, чтобы не 
только сделаться собственным антикваром, хотя бы  
критически, но и чтобы книжную пыль моего прошлого 
употребить вместо с тем в качестве удобрения для 
44 
окончательного завершения моей темы, хотя бы в  
основных чертах. В связи с этим я и начинаю с  
настоящего тома, исходя из его основного содержания,  
названного «Пояснениями и дополнениями к «Сущности  
христианства»»; дополнения эти содержат в себе как  
наиболее существенные выводы, так и предпосылки  
«Сущности христианства». 


\end{document}
